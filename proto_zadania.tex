\documentclass{article}
\usepackage{polski}
\usepackage[utf8]{inputenc}
\begin{document}
\section{Termodynamika}
\subsection{2 zasada termodynamiki}
\section{Dynamika}
\subsection{Równia pochyła}
Zadanie 1 \\
Cząstka $\alpha$ wpada w pole magnetyczne o indukcji B = 0,02 T prostopadle do kierunku wektora indukcji $\rm \vec{B}$
i zatacza krąg o promieniu r = 0,2 m. Oblicz energię cząstki $\alpha$ i wyraź ją w keV. 
Masa cząstki $\alpha$ jest znana i wynosi 4,0027 a.j.m.
Rozwiązanie \\

\end{document}