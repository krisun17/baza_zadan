Tor zbudowany z płaskiej i gładkiej stalowej taśmy ma kształt okręgu o promieniu \emph{R} i ustawiony jest w~płaszczyźnie pionowej. Ciało o masie \emph{m} porusza się ze stałą wartością prędkości liniowej \emph{v} po wewnętrznej stronie toru. Zbadać, jaka siła działa na tor, gdy ciało znajduje się w punktach \emph{A} i \emph{B} na wysokościach \emph{h} oraz $2R - h$, gdzie $h < R$.

\begin{figure}[H]
	\centering
	\includegraphics[width=0.3\linewidth]{../rysunki/dynamika/petla-klocki-nacisk-dwa-punkty.png}
\end{figure}

%poziom 2