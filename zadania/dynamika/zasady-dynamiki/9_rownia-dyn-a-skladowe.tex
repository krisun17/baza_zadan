Klocek o masie \emph{m} leży na wadze sprężynowej ustawionej na podstawie \emph{A}, jak na rysunku. Podstawa ześlizguje się z równi pochyłej o kącie nachylenia \emph{$\alpha$}. Jaką siłę wskaże waga sprężynowa, jeżeli ruch po równi odbywa się bez tarcia? Jaki powinien być współczynnik tarcia \emph{$\mu$} między klockiem i szalką wagi, aby klocek nie ześlizgnął się z niej?

\begin{figure}[H]
	\centering
	\includegraphics[width=0.3\linewidth]{../rysunki/dynamika/rownia-dyn-a-skladowe.png}
\end{figure}

%poziom 3