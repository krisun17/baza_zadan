Dwa ciała o masach \emph{$m_1$} i \emph{$m_2$} połączono nicią, która jest przerzucona przez bloczek znajdujący się w wierzchołku równi o kącie nachylenia \emph{$\alpha$}. Współczynnik tarcia między ciałem o masie \emph{$m_2$} i równią wynosi \emph{$\mu$}. Masę bloczka zaniedbać. Jaka powinna być masa \emph{$m_1$}, aby ciało o masie \emph{$m_2$} poruszało się: a) w górę równi, b) w dół równi?

\begin{figure}[H]
	\centering
	\includegraphics[width=0.3\linewidth]{../rysunki/dynamika/klocki-rownia-jakie-m.png}
\end{figure}

%poziom 2