Na samochód o masie \emph{m} = 1000 kg, poruszający się z prędkością \emph{$v_0$} = 72 km/h po prostym torze poziomym, w pewnej chwili zaczęła działać siła hamująca \textbf{\emph{F}}, skierowana przeciwnie do ruchu. Jaka jest wartość siły \textbf{\emph{F}}, jeżeli samochód zatrzymał się po upływie \emph{t} = 5~? Jaką odległość \emph{s} przejedzie samochód do chwili zatrzymania się?
%Kruczek