Łódź płynie z prądem rzeki z przystani \emph{A} do \emph{B} w czasie \emph{$t_1$}~=~3~godz, a z \emph{B} do \emph{A} w czasie \emph{$t_2$}~=~6~godz. Ile czasu trzeba, aby łódź spłynęła z przystani \emph{A} do \emph{B} z wyłączonym silnikiem?

%Kruczek 1-10R/15 
%Poziom A
