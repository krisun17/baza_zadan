Pociąg \emph{A} ma długość \emph{$s_A$}, pociąg \emph{B} długość \emph{$s_B$}. Gdy pociągi się mijają jadąc w tę samą stronę, to czas, który upływa od chwili gdy lokomotywa \emph{A} dogoni ostatni wagon pociągu \emph{B} do chwili gdy ostatni wagon pociągu \emph{$s_A$} minie lokomotywę B, wynosi \emph{$t_1$}. Gdy pociągi jadą w przeciwne strony, czas mijania wynosi \emph{$t_2$}. Obliczyć prędkości \emph{$v_A$} i \emph{$v_B$} obu pociągów.

%Kruczek 1-2/13 
%Poziom A